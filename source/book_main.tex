%%
%% Date: 22 Mar 2022
%%
%%
\documentclass[12pt]{book}
%%
\usepackage{xspace}
\usepackage{url}
%%
\usepackage{graphicx}
%%
\usepackage{amsmath}
\usepackage{amssymb}
\usepackage{amsthm}
\usepackage{thmtools}
\usepackage[all,cmtip]{xy}
%%
\usepackage[framemethod=tikz]{mdframed}
\usepackage{listings}
%%
\usepackage{xcolor}
%%
\usepackage{makeidx}
%%
\graphicspath{{./figures}}
%%
\makeindex
%%
%% ----------------------------------------------------------------------
\begin{document}


%% ----------------------------------------------------------------------
\chapter{Outline}

\section{Introduction}

\begin{itemize}
\item Overview of what the book is about;
\item Relevance: for systems that are (or are in need of) being
  analyzed nowadays (networked);
\item Gap in the literature;
\item Prepare list of related literature, e.g.,~\cite{Law:07}. This
  may be somewhat tricky, so it may useful to state something about
  what we choose to focus on while acknowledging examples that we do
  not address (possible example: Petri Nets type work);
\item Contrast with ``continuous'' modeling and analysis; make the
  point that models that are ``continuous'' may also be part of what
  we are covering;
\item Book companion web site;
\item Aspects that will help adoption for courses;
\end{itemize}

%% ----------------------------------------------------------------------
\section{Examples of (real) systems}

\begin{itemize}
\item An example to be used throughout the book;
\item NPS-1;
\end{itemize}

%% ----------------------------------------------------------------------
\section{Formal modeling framework}

\begin{itemize}
\item Why?
\item The ``implementation defines the model'' pitfall
\item Graph dynamical systems
  \begin{itemize}
  \item Why GDS? As a minimum, provides a precise + conceptual
    framework suited for modeling networked systems
  \item Analysis types: mathematical, computational, numerical,
    game-theoretic
  \end{itemize}
\item Validation and verification; sensitivity analysis;
\end{itemize}

%% ----------------------------------------------------------------------
\section{Synthetic populations and digital twins}

\begin{itemize}
\item Why?
\item Examples:
\item How?
\end{itemize}

%% ----------------------------------------------------------------------
\section{Unencapsulated Agency}

\begin{itemize}
\item Central idea; coordinate system(s);
\item Examples;
\item System factorization/decomposition; validity;
\end{itemize}

%% ----------------------------------------------------------------------
\section{Architecture design}

\index{architecture}

\begin{itemize}
\item What is it? Mapping models to software (aka modules);
\item The fact that this is a non-trivial but central aspect;
\item Construction of modules from system components;
\item Coordinating architecture and module contracts; integration of metrics;
\item Computing environment;
\item Designing for scaling; (what does the scenario/application
  require in terms of scale?)
\item Data management; (e.g., communcation among modules; provenance)
\item Scientific reproducibility;
\item Verfication and validation; integration of tests within framework;
\item System testing harnesses;
\item SciDuct;
\end{itemize}

%% ----------------------------------------------------------------------
\chapter{Introduction}

Starts here ...


%% ----------------------------------------------------------------------
\bibliographystyle{plain}
\bibliography{references}
%% ----------------------------------------------------------------------
\printindex
%% ----------------------------------------------------------------------
\end{document}
